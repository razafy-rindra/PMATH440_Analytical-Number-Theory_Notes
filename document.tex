\documentclass{article}
\usepackage[margin=0.5in]{geometry}
\usepackage[utf8]{inputenc}

\usepackage{amsmath}
\usepackage{amsthm}
\usepackage{amssymb}
\usepackage{enumerate}
\usepackage{chngcntr}
\usepackage{mathtools}


\newcommand{\Z}{\mathbb{Z}}
\newcommand{\C}{\mathbb{C}}
\newcommand{\HH}{\mathbb{H}}
\newcommand{\Q}{\mathbb{Q}}
\newcommand{\R}{\mathbb{R}}
\newcommand{\N}{\mathbb{N}}
\DeclarePairedDelimiter\ceil{\lceil}{\rceil}
\DeclarePairedDelimiter\floor{\lfloor}{\rfloor}

\usepackage[dvipsnames]{xcolor}

\newtheorem{theorem}{Theorem}[section] 
\newtheorem{corollary}{Corollary}[theorem] 
\newtheorem{lemma}[theorem]{Lemma} 
\newtheorem{proposition}{Proposition}[section]

\theoremstyle{definition}
\newtheorem{definition}{Definition}[section]
\theoremstyle{remark}
\newtheorem*{remark}{Remark}
\theoremstyle{definition}
\newtheorem{example}{Example}[definition]
\newcounter{exercise}[subsection]
\newenvironment{exercise}{\refstepcounter{exercise}\textbf{Exercise~\theexercise}\begin{proof}}{\end{proof}}
\counterwithin*{equation}{section}
\counterwithin*{equation}{subsection}



\title{Analytical Number Theory}
\author{Razafy Rindra}
\date{Fall 2019}
\begin{document}
	\maketitle
	\section{Notation}
		\begin{definition}
			Landau's Big-Oh, Little-Oh and $\sim$. Let $a$ be finite or infinite:
			\begin{enumerate}
				\item We say that $f(x)\sim g(x)$ as a $x\rightarrow a$, if\begin{equation*}
					\lim_{x\rightarrow a} \frac{f(x)}{g(x)} = 1
				\end{equation*}
				\item Big-oh: We say that $f(x) = O(g(x))$ as $x\rightarrow \infty$, if there exists $x_0\in \R$, and $c>0$, such that:\begin{equation*}
					|f(x)| \leq cg(x) \ \text{ for all } x>x_0
				\end{equation*}\\
				 We say that $f(x) = O(g(x))$ as $x\rightarrow a$, if there exists $\delta>0$, and $c>0$, such that:\begin{equation*}
				|f(x)| \leq cg(x) \ \text{ for all } |x-a|<\delta
				\end{equation*}
				\item Little-oh: We say that $f(x) = o(g(x))$ as $x\rightarrow a$, if\begin{equation*}
				\lim_{x\rightarrow a} \frac{f(x)}{g(x)} = 0
				\end{equation*}
			\end{enumerate}
		\end{definition}
	\begin{example} For $\sim$:
			\begin{enumerate}
				\item $\sin(x)\sim x$, as $x\rightarrow 0$.
				\item Stirlings:\\
					$n! \sim (\frac{n}{e})^n\sqrt{2\pi n}$, as $n\rightarrow \infty$.
			\end{enumerate}
	\end{example}
	\begin{example} For Big-Oh:
		\begin{enumerate}
			\item $\sin(x) = O(1)$. As $x\rightarrow \infty$ since $\sin(x)$ is bounded.
			\item $\sin(x) = x - \frac{x^3}{3!} + O(x^5)$. As $x\rightarrow 0$ since: \begin{equation*}
				\sin(x) = \sum_{n=0}^{\infty} \frac{(-1)^{n}x^{2n+1}}{(2n+1!)} \Rightarrow \sin(x) - x - \frac{x^3}{3!} = \frac{x^5}{5!} - \frac{x^7}{7!} +\cdots
			\end{equation*}
		\end{enumerate}
	\end{example}
	\begin{example} For Little-Oh: $n! = o(n^n)$, since \begin{equation*}
			\frac{n!}{n^n} \sim \frac{\sqrt{2\pi n}}{e^n} = \sqrt{2\pi}e^{\frac{1}{2}\log(n) - n}
		\end{equation*}
	\end{example}
\section{Summation by parts}
Let $f$ be a function from $\Z^+$ to $\R$ or $\C$, and $g$ a real or complex valued function of a real variable. If $g'$ exists and is continuous on $[1,x]$ for some $x\in R$. We find that:\begin{equation}
	\sum_{1\leq n \leq x} f(n)g(n) = (\sum_{1\leq n\leq x}f(n))g(x) - \int_1^x \sum_{1\leq n\leq t}f(t)g'(t)dt
\end{equation}
\begin{proof}
	\begin{align*}
		f(n)g(n) &= f(n)g(x) - f(n)\int_{n}^x g'(t)dt\\
		\sum_{1\leq n\leq x}f(n)g(n)&= \sum_{1\leq n\leq x}f(n)g(x) - \sum_{1\leq n\leq x}f(n)\int_{n}^x g'(t)dt\\
	\end{align*}
	We can bring in all the $f(n)$ where $n\leq t$, inside the same integral, and this gives us our result.
\end{proof}
\begin{proof}
	Alternate proof using telescoping:
		Since $\int_1^x = \int_1^2 + \int_2^3+\cdots + \int_{\floor*{x}}^x$
	  \begin{align*}
	  	&\sum_{1\leq n\leq x}f(n)g(x) - \int_1^x (\sum_{1\leq n \leq t}f(n))g'(t)dt = \\
	  	&(\sum_{1\leq n\leq x}f(n))g(x) - f(1)(g(2)-g(1)) - (f(1) + f(2))(g(3) - g(2)) - \cdots - (\sum_{1\leq n\leq x}f(n))(g(x) - g(\floor{x}))
	  \end{align*}
	  By telescoping we get the desired result.
\end{proof}
\subsection{$\sum_{1\leq n\leq \floor*{x}}$}

	$\sum_{1\leq n\leq \floor*{x}} \frac{1}{n}$, in this situation we have $f(n) = 1$ and $g(x) = \frac{1}{x}$. So, we have: $\sum_{1\leq n \leq x} 1 = \floor{x}$. We define ${x}$ to be the fractional part of $x$, so that $\floor{x} = x - \{x\}$.
	
	Hence we have:\begin{align*}
		\sum_{1\leq n\leq \floor*{x}} \frac{1}{n} &= \floor{x}\frac{1}{x} + \int_{1}^x \floor{t}\frac{1}{t^2}dt\\
		&= 1 - \frac{\{x\}}{x} + \log(x) - \int_{1}^x\frac{\{t\}}{t^2}dt\\
		&= 1 - \frac{\{x\}}{x} + \log(x) - (\int_{1}^\infty\frac{\{t\}}{t^2}dt- \int_{x}^\infty\frac{\{t\}}{t^2}dt)\\
	\end{align*}
	Since $- \frac{\{x\}}{x} = O(\frac{1}{x})$ and $\int_{x}^\infty\frac{\{t\}}{t^2}dt) = O(\frac{1}{x})$, we finally get at the end:\begin{equation*}
		\sum_{1\leq n\leq x}\frac{1}{n} = \log(x) + (1 - \int_{1}^\infty\frac{\{t\}}{t^2}dt) + O(\frac{1}{x})
	\end{equation*}
	We call $1 - \int_{1}^\infty\frac{\{t\}}{t^2}dt = \gamma = 0.5772\cdots$ Euler's constant.


This example tells us that: $$\lim_{x\rightarrow\infty} (\sum_{1\leq n\leq x}\frac{1}{n} - \log(x)) = \gamma$$

\

\subsection{$\log(m!)$}
Consider: $\log(m!) = \sum_{1\leq n\leq m}\log(n)$; let $f(n) = 1$ and $g(x) = \log(x)$.\begin{align*}
	\log(m!) &= m\log(m) - \int_{1}^m \frac{\floor{t}}{t}dt\\
	&= m\log(m) - \int_{1}^m \frac{t-\{t\}}{t}dt\\
	&= m\log(m) - (m-1) + \int_{1}^m \frac{\{t\}}{t}dt\\
\end{align*}

Since $0 < \int_{1}^m \frac{\{t\}}{t}dt <\log(m)$, we have $m\log(m) - (m-1) < \log(m!) < (m+1)\log(m) - (m-1)$\\
Thus: $$\frac{m^m}{e^{m-1}}< m! <\frac{m^{m+1}}{e^{m-1}} \text{ in reality }m!\sim (\frac{m}{e})^m\sqrt{2\pi m}$$
\end{document}